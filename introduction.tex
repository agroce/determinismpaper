\section{Introduction}

For ourselves, we might prefer to think (and act as if) we had free
will; however, we generally prefer our software systems to be as
constrained in their actions as possible:  in other words, we wish
them to be largely deterministic, from our perspective.

Determinism is particularly important for testing and debugging, where
being able to exactly reproduce system behavior is essential to
productivity \cite{Firesmith}.  If executing ``the same test'' can produce
significantly different behavior each time it is run, the consequences
can be unfortunate.  Developers using a test exhibiting nondeterminism
to debug a system
face a challenge in reasoning about causality, in that an event
observed may not even take place the next time the test is run; this
pernicious phenomenon has been sometimes referred to as the dreaded
``Heisenbug'' \cite{Heisenbug}, though the more accurate usage is
``Mandelbug'' \cite{GrottkeBugs,FaultTriggers}, since Heisenbugs, properly speaking, are bugs that disappear or
alter behavior under instrumentation (especially in debugging).
Automated test generation systems using test coverage results to drive
the search for interesting inputs may be mislead when a run executes
code only rarely.  And, most significantly, regression testing
effectiveness can be significantly reduced if tests fail only
intermittently and unpredictably, as a result of environmental
factors rather than bugs in changed code.  Such behavior is
unfortunately all too common:  Gao et
al. \cite{Gao:2015:MSU:2818754.2818764} observed coverage differences
of up to 184 lines of code for the same test, and false positive rates
as high as 96\%.  Nondeterminism is sometimes problematic for
developers, who will usually want to  assume that library code they
use behaves in a reliably predictable fashion; nondeterminism is
frequently vexing in debugging efforts; most importantly, perhaps, nondeterminism is \emph{often disastrous} for large-scale highly
automated testing.


\subsection{What is Determinism?}

A system is deterministic if, given the system's complete state at a point in
time, it is possible, in principle, to predict its future behavior
perfectly.  We say ``in principle'' because in the real world,
prediction may be possible but hopelessly impractical.  We write
complex software systems in many cases because we cannot predict their
behavior (if we could perform the calculations in advance, ourselves,
we would just do so).  As a consequence, in software, rather than
defining determinism in terms of prediction, we usually therefore
simply say that a system is deterministic if, given the same state and
inputs, it always produces the same outputs.  

Technically, many ``nondeterministic'' systems are not
nondeterministic in a strong sense, at all.  Given the complete state
of the system (which includes the entire state of the underlying
hardware, the operating system, storage devices, etc.), ignoring
quantum effects, and treating outside interventions such as network
traffic, human activity at an input device, etc. properly as inputs,
most software \emph{is} completely predictable.  What we actually
mean, usually, is that, given a certain limited abstraction of state
and of inputs, observable behavior is repeatable.  This abstraction,
for most software, is not expected to include many elements outside of
the software system itself.

Because the
programmer's abstraction of the system ignores  such a large number of
details, very few tests are in fact ``deterministic'' in the sense
that they eliminate all changes in behavior between executions.  Running the same test
twice almost always results in differences, given a low enough level of
abstraction, since the system load,
cache contents, branch predictor history, etc. are almost never
controlled for; however, this kind of
nondeterminism is usually of no interest, unless the test involves
extremely precise timing constraints.  Rather, what matters is when
some kind of nondeterminism unexpectedly impacts computed values at a
higher level of abstraction; in general, if the operating system
itself is not buggy, low-level nondeterminism, by design, is invisible
except in fine-grained performance testing or real-time systems.
Unexpected nondeterminism usually arises when there is an element of higher-level
state or input that \emph{is} critical to the produced behavior, but
the programmer has not anticipated.  E.g.,  when it is believed that the
behavior of a thread scheduler will not matter, but a race condition
in the code means that it does matter, or when the order of items in
an iterator on a hash table is important, and the hash values used are
randomly salted.

\subsection{The High Cost of Unexpected Nondeterminism}

Unexpected nondeterminism is, unfortunately, usually only discovered
in a context that makes it very hard to debug.   The most common such
contexts are occasional rare failures of a system in deployment, and
regression tests that do not behave reliably (known as flaky tests).
Furthermore, unexpected nondeterminism makes it difficult to use
automated test generation to produce effective regression tests for a
system.  While developers may know how to produce reliably
deterministic unit tests, even in the presence of underlying
nondeterminism, automated test generation, without a large investment
in human time, usually does not have sufficient information to avoid
producing the occasional such test.  Moreover, where a human might
make an assertion about the final value produced during a unit test,
an automatically generated regression test is usually most easily produced by
simply asserting equality with all values produced during a reference
run, since the final step of a test
is likely to be somewhat arbitrary, generated in an effort to increase
code coverage, not establish some functional correctness property.
This may be one reason that automated testing is seldom used to
produce general regression tests.  While failure-inducing tests
produced automatically are often added to regression suites, the full
set of passing tests is, to our knowledge, only infrequently preserved
as part of a typical regression suite, perhaps because such tests are
likely to be \emph{flaky} without considerable human effort.

\subsubsection{Nondeterminism and Flaky Tests}

In order to help ensure that they are reliable and secure, complex
modern software systems usually include a large set of
\emph{regression tests}.  A regression test suite is a (usually large)
set of tests that can be run against a software system every time it
is modified, to ensure that the modification has not broken the system
in some way.  \emph{Flaky tests} \cite{miccoflaky} are regression
tests that fail in an intermittent, unreliable fashion, and thus
degrade the utility of regression testing.  The essence of a flaky
test is that, for the same snapshot of test code and code under test
(CUT), it sometimes fails and sometimes passes: the pass/fail result (disposition) of the
test is not a deterministic property of the test code, code under
test, and environment in which the test is run\footnote{Again, in 
  some sense the result \emph{is} clearly determined by the environment in
  which the test is run, but from the point of view of a developer or
  test engineer, the differences in environment are invisible and unknown.}.  This produces three serious
problems: first, a flaky test often wastes developer time and delays
software changes by forcing the investigation of code-under-test that
is not actually incorrect.  Second, failures in flaky tests (for that
reason) are often ignored, and therefore serious software faults may
be missed.  Finally, to mitigate these problems, flaky tests are often
run multiple times, wasting valuable computing resources and also
delaying acceptance of code changes.  An analogy may make the extent
of the problem in large-scale test automation clear:  a canary in a coal mine is of
little use if canaries frequently become ill for reasons unrelated to
the presence of toxic gases.  Mining may stop for no good reason, or
miners may learn to ignore the canary, leading to tragedy; a third,
more ``practical'' option is that miners may carry so many redundant
canaries into the coal mine that canary-care becomes a serious burden
on mining.

Flaky tests are, for us, simply a special case of general test
nondeterminism, where the nondeterminism of test values is sufficient
to cause the pass/fail result of the test to vary.  The definitions
and techniques proposed in this paper all apply to flaky tests, as a
special case of various kinds of \emph{horizontal} nondeterminism
described below.  Our focus, however, unlike all other tools and
approaches we are aware of, is on detecting the \emph{sources} of
flaky-ness \emph{before} they propagate to cause actual test failure.

\subsection{Contributions:  Detecting and Debugging Nondeterminism}


This paper proposes (1) a number of formal definitions of types of
nondeterminism (\emph{horizontal} and \emph{vertical}) (see Section \ref{sec:formal}) and (2) an implementation, based on these definitions, for detecting and debugging
nondeterminism in property-based testing.  The implementation is based
on an approach where (3) horizontal determinism is considered as a
kind of \emph{reflexive differential testing} (4) vertical
determinism is specialized to the common case of \emph{failure
  determinism}, and (5)  in both cases the formalism is made practical
by using the \emph{value pool} model of unit tests to define a useful
granularity for nondeterminism detection (see Section \ref{sec:pools}).
We further propose modifications to the widely used delta-debugging algorithm in order to better handle
nondeterminism as a test property.

The underlying idea in this paper is that most approaches to
nondeterminism/flaky test detection and debugging rely either on 1) identifying
potential sources of nondeterminism such as test inter-dependence
\cite{LamZE2015} or certain code smells \cite{palomba2017does}  or 2)
use a heuristic approach to mark failures as likely flaky, such as that taken by DeFlaker
\cite{bell2018d}.  In the real world, people usually detect flaky
tests by the most obvious heuristic of all:  they observe a test both
fail and pass for the same code version.  In a sense, even DeFlaker
relies on this method, but avoids having to observe a flaky test actually
passing, by inferring that if a test was previously passing, and
executes no changed code, its failure after a code change \emph{must}
be due to flakiness/nondeterminism.  Our approach to the problem is
fundamentally different, and essentially orthogonal; we concentrate on the detection and avoidance of
nondeterminism/flakiness using \emph{automated test generation},
and make it possible for a developer or test engineer to detect when
important values generated in automated testing differ without a
change in the test itself or the code under test.  While some of these
changes may be harmless, and unable to propagate to cause test
failures, they are potential sources of flaky behavior.  To our
knowledge, this is the first approach to the problem that
allows detection of (potential sources of) flakiness, based on real divergence in
behavior, without having to observe a test actually fail due to nondeterminism.

This approach enables a number of useful results:
 the automated production of
nondeterminism-free regression tests even for libraries with some
sources of nondeterminism, the documentation of sources of
nondeterminism that might produce problems for manually-constructed tests, and, most importantly, detection and effective debugging for
unexpected nondeterminism that is, itself, a bug.  Additonally, by
identifying (using automated testing) library behaviors that are
nondeterministic, our approach should make it much easier to flag
statements in existing human-written unit tests that may produce flaky
behavior:  simply scan the
code for calls to functions identified as sources of nondeterminism.

We use a set
of case studies. including real-world, widely-used, Python libraries to demonstrate the
utility of our ideas.  Failure determinism directly
produces a more than 5\% improvement in mutation score for an already highly
effective automated test generation tool for a file system, with no
additional specification or test design burden.