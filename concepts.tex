\section{Practical Lightweight Nondeterminism
  Detection and Debugging}



\subsection{Horizontal Determinism}

\subsubsection{Determinism and Reflexive Differential Testing}

Horizontal determinism can be best understood by thinking of
nondeterminism detection as an unusual kind of \emph{differential
  testing} \cite{Differential,ICSEDiff}.  In differential testing, a
system is compared against a reference in order to ensure that
it behaves equivalently, at some level of abstraction, to another
implementation of the same functionality.  Differential testing is
extremely powerful, in that any divergence of behavior, if divergence
is correctly defined, indicates a functional correctness fault in (at
least) one of the systems under test.  Being able to detect functional
correctness errors without the cost of constructing a formal
specification is extremely useful in automated test generation.  Differential testing is widely
used for systems software components such as compilers
\cite{Differential,csmith} and POSIX file systems \cite{CFV08,AMAI}, where multiple
implementations are common.  The major limitation of differential
testing, of course, is that multiple implementations of a system are
almost as rare as good correcteness specifications.

For the special case of detecting nondeterminism, however, \emph{a
  system can serve as its own reference implementation.}  

\subsubsection{Visible Value Determinism}

\subsubsection{Final State Determinism}

\subsubsection{Process-Level Determinism}

\subsection{Vertical Determinism}

Apple bugs.

\subsubsection{Failure Nondeterminism}

\subsection{Nondeterminism and Delta-Debugging}

Delta-debugging \cite{DD}\footnote{We use delta-debugging to stand for
  all test reduction algorithms, even those \cite{CReduce,tstl} that do not use the exact
  binary-search that distinguishes delta-debugging \emph{per se}; the
  differences are immaterial for our purposes.}  is a widely used method for reducing the
size of failing tests, making them easier to understand and debug.
Delta-debugging in the context of detecting nondeterminism has two
purposes.  One is simply the usual goal of reducing the size of a
test.  Identifying the cause of nondeterminism may be very easy in a
test consisting of ten library function calls (it is one of these ten
calls), but very difficult in a test consisting of a hundred library
function calls.  This is no different than the common use of
delta-debugging.  However, in nondeterminism detection,
delta-debugging can also be used to \emph{change the probability of
  nondeterministic behavior}.

In \emph{monotonic} cases, removing a part of a test cannot
increase the probability of a predicate holding.  This is fairly
common.  In these cases, when we use a reduction algorithm to reduce $t$
to $r$, $P(p(r) \leq P(p(t))$.  In \emph{non-monotonic} cases,
however, there is no such upper bound.  Removing a step in $t$ may
increase the probability that $t$ behaves nondeterministically.

The probabilistic predicate of interest for determinism is always of
the form:  {\bf the test will exhibit at least two different behaviors in $S$
runs, with probability $P$}.  This predicate may be monotonic or
non-monotonic, depending on the causes of the nondeterminism detected.

The only absolutely necessary changes required to use standard
delta-debugging implementations in nondeterminism detection
are simply removal of some ``sanity checks'' in the code:  many
implementations (including the Python code provided by Zeller) assert
that the predicate holds on the original test.  In probabilistic
settings, this may not be true, and we may even be trying to find a
subset where a predicate that is very far from holding on the original test
holds (in a non-monotonic case where we aim to increase the probability
of nondeterminism).

\subsubsection{``Publication Bias'' in Delta-Debugging}

However, simply using delta-debugging off the shelf with a predicate
like $P(fail) > 0.3 \wedge P(fail) < 1.0$, to force a test to be flaky, and force it to fail
sufficiently often to be used in debugging, will often produce
surprising and unfortunate results.  In many cases,
delta-debugging will indeed reduce the large test to a small subset.
And, in a technical sense, delta-debugging will work:  it will never
convert a nondeterministic test to a completely deterministic test,
because the reduced test that delta-debugging returns is always one
where the predicate of interest has been seen to evaluate to true.
However, if you run the resulting test, it will, in many cases, have a
$P(fail)$ that is much, much smaller than
0.3, perhaps as low as 0.01.  Why?

The problem is analogous to the problem of publication bias in
scientific fields \cite{ahmed2012assessment}.  A predicate like
$P(fail) > 0.3 \wedge P(fail) < 1.0$ cannot be evaluated by
determining the true probability of failure; rather, the test must be
run some concrete number of times, and the number of failures counted.
However, even if the number of samples $N$ is large, there is some
probability (based on the sample size) of a result that diverges
significantly from the actual probability of failure.  If the
predicate were run only once, and the number of samples reasonably
large, this would not matter.  However, by their nature,
delta-debugging and other test reduction algorithms, explore a search
space that often contains hundreds or even thousands or millions of
candidate tests.  The predicate is evaluated on each of these, and so
even with large $N$, it is extremely likely that some evaluation will
produce a very poor estimate of $P(fail)$.  If such an evaluation
causes the predicate to appear to hold for a test, delta-debugging is
``stuck'' with the error, because (to our knowledge) no test reduction
algorithms allow backtracking.  After such a mistake, finding further
reductions will become harder, but may still be possible due to the
same source of errors:  the number of experiments run is far larger
than the probability of an incorrect result.

It is the combination of
a one-way bias on faulty evaluations (the consequences of a false
positive for the predicate are much greater than for a false negative)
and the huge number of experiments relative to error rate, that
produces bad results, akin to the magnification of effect sizes
in science due to publication bias.  The bias in delta-debugging tends
towards producing a reduced test where probabilities are much smaller
than demanded by a predicate, due to the pressure on delta-debugging
to produce shorter tests.  Shorter tests have smaller probabilities,
on average, due to two factors:  first, timing-induced nondeterminism
has a much smaller temporal space to operate in, and second, if some
test operation introduces a small probability of nondeterminism, and
only many repetitions of that operation make the probability large,
small tests obviously on average have fewer instances of the
problematic operation.

In scientific literature, the most frequently proposed solution is the
use of replications:  repeated runs of ``successful'' experiments to
minimize the probability that a result is a fluke due to publication
bias.  One way to produce this effect would be to allow
delta-debugging to backtrack if the probabilties observed in predicate
evaluations suddely exhibit a strong discontinuity, a kind of ``paper
retraction'' based on near-replications.  However, this requires
modifying the delta-debugging implementations, which is difficult and
sometimes not really feasible (e.g., few users of AFL-fuzz \cite{aflfuzz} will wish
to change its C code for test reduction).  Ideally, the solution
should be implementable simply by modifying the predictate that is
evaluated.  

A costly but effective solution is to make $N$ large in comparison to
the number of expected predicate evaluations performed during
delta-debugging.  If the error rate is low enough, the problem
disappears.  However, given the large number of evaluations performed,
this will tend to make delta-debugging extremely slow.  We propose
using a dynamic sampling approach, where $N$ is small, but if the
predicate evaluates to true, $M$ repeated true evaluations
(``replications'') are
required before the predicate returns ``true'' to the delta-debugging
algorithm.  Tuning $M$, any degree of confidence can be achieved, with
the basic tradeoff being between finding a test with the desired
probabilities, and the speed and effectiveness of delta-debugging.
Because increasing $M$ makes false negatives more likely, larger $M$
will usually result in less-than-optimal reduction of the original test.