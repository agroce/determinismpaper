A critically important, but surprisingly neglected in the software testing literature, aspect of system reliability is system predictability.  Many software systems are implemented using mechanisms (unsafe languages, multiple threads, complex caching, stochastic algorithms, environmental dependencies) that can introduce behavioral nondeterminism.  Users of software systems, especially other software systems using library calls in a single-threaded context, often expect that systems will behave deterministically in the sense that they have predictable results from the same sequential series of operations.  Equally importantly, even when it does not impact correctness or user experience, nondeterminism must be understood and managed for effective (especially automated) testing.  Nondeterministic behavior that is not either controlled or accounted for can result in flaky tests as well as causing problems for test reduction, differential testing, and automated regression test generation.  We show that lightweight techniques, requiring (almost) no effort on the part of a developer, can be used to extend an existing test generation system to allow checking for problematic nondeterminism.  We introduce a set of lightweight nondeterminism properties, inspired by real faults, and tailor these notions to the practical, automatic, checking of Python library code.  In particular, we demonstrate that our proposed failure nondeterminism can improve mutation score by 6\% for a strong, full differential test harness for a widely used mock file system. 