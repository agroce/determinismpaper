\section{Experimental Evaluation}

\subsection {Case Study: {\tt redis-py}}

The {\tt redis-py} \cite{redispy} module implements a widely-used Python interface
to the popular Redis in-memory database, cache, and
message broker \cite{redis}.

Using TSTL's harness for {\tt redis-py}, both {\tt redis-py} and Redis
itself can be tested; unfortunately, generating standalone
high-coverage regression tests for {\tt redis-py} and Redis using TSTL
was difficult, as numerous Redis commands introduce nondeterministic
behavior.  Some of these are obvious (e.g., {\tt randomkey, srandmember}) by
name.  Given an understanding of Redis semantics, that the various
commands producing data with a limited lifetime (e.g., {\tt expire,
  pexpire}) also introduce timing-based nondeterminism is also clear.
However, that a command such as {\tt restore} takes an expiration
argument is not obvious on inspection, nor is the behavior of {\tt
  spop} which pops a random value from a set.  Moreover, it is
difficult to guess whetehr the {\tt pipe} mechanism, which allows a
large sequence of commands to be queued up in a pipe and executed all
at once, introduced the potential for nondeterminism (does it execute
sequentially before other command are handled, or is it in parallel
with further commands).  Deep experience with Redis would make these
issues clear, but tests using a library are often written by those not
intimately familliar with the semantics of every used library (otherwise they would not have
been as likely to introduce flaky tests in the first place).  This is
even more true in the case of automated test generation, where the
test engineer is often chosen for expertise in test generation tools,
not in the domain of the software under test (SUT), and is seldom the
original developer of the code.

Using the original TSTL harness, approxoimately \emph{over 20\%} of all {\tt redis-py}
regression tests (of length 200) generated were flaky.  Using the
nondeterminism checker to reduce flaky tests to a minimal set allows
an engineer to leave an overnight run that will, in 12 hours, produce
a set of minimal flaky tests exposing the sources of
nondeterminism/flakiness.  In this case, the probability of flakiness
is not important, just the possibility, so we let the delta-debugging
create a minimal test with some non-zero (but possibly very small)
observed probability of nondeterminism.  

After removing all identified sources of nondeterminism (the {\tt random-} functions, all
expiration-related functionality and {\tt restore} calls with non-zero
lifetimes, and {\tt spop}), but leaving all other suspect funcitonality in
place (e.g. pipes), no flakiness was observed in a sample of over 1,000 length
200 tests.  Moreover, because the removals were limited to acutally
observed sources of flakiness, the code coverage loss was minimal.
Mean branch coverage, for regression tests of length 200 was reduced
by less than 5 branches (a less than 1\% decrease).  Obviously, it is
impossible to test the removed calls, but the overall coverage loss is
both minimal and known, and can be made up for using specially-crafted
tests (for instance, wrapping the problematic calls in a way that does
not check the values, or adding a delay after an expiration to allow
the data to expire).  As a price to pay for the ability to produce
fast-executing high-coverage full regression tests (not just tests for
crashes and unexpected exceptions), this is a very acceptable price.

Moreover, turning off sources of expected nondeterminism makes
it possible to aggressively test {\tt redis-py} and Redis for
unexpected nondeterminism arising from actual bugs.  The overhead for such
determinism checks, with no delay between operations, is only
20\%, much lower than the expected cost of running each test twice.
This is because choosing the actions in a test (and determining which
actions are enabled at each step) consumes a large part of the test
generator's time; running the test again and checking equality is
relatively inexpensive.  

\subsection{Case Study: {\tt datarray} Inference Algorithms}

The {\tt datarray} module \cite{datarray} is a prototype
implementation for numpy arrays with named axes to improve data
management, developed by the Berkeley Institute for Data Science.  As part of its code, it provides a set of algorithms for
inference in
Bayesian belief networks \cite{russell2016artificial}.  An earlier
version of these algorithms produced nondeterministic (and in some
cases incorrect) results due to dependence on the order of values in
an iterator over a Python dictionary (see Section \ref{sec:pnondet}).

Figure \ref{hashbug} shows TSTL code for generating inputs to the {\tt
  datarray} algorithms.  Running this harness using TSTL's {\tt
  --checkProcessDeterminism} flag required less than 10 seconds on
average to produce a test exhibiting process-level nondeterminism in
the {\tt calc\_marginals\_sumproduct} function (the only broken
algorithm).  Reducing this 60 step test to a minimal test of only 6 steps,
showing an extremely simple input producing the issue, 
required another 92 seconds.

Removing the nondeterministic call, we can see that the cost of
checking for process nondeterminism, with no delay between operations,
is high, a 93\% slowdown.  This is due to the very high cost of
using a new subprocess to re-run the test, and comparing values using
text I/O on output pipes, rather than on values alone.


\begin{figure*}
{\scriptsize
\begin{code}
@import inference\_algs
@import datarray
\vspace{0.1in}
<@
def flatten\_and\_sort(v):
    return (sorted(map(flatten\_and\_sort,v),key=repr) if type(v) in [list,tuple] else
                (flatten\_and\_sort(list(v.items())) if type(v) == dict else v))

def psplit(P):
    return ([P,1.0-P])
@>
\vspace{0.1in}
pool: <P> 3
pool: <cpts> 3
pool: <evidence> 3 OPAQUE
pool: <ename> 3
pool: <event> 3
\vspace{0.1in}
<P> := 0.01 * <[0..100]>
\vspace{0.1in}
<ename> := "E" + str(<[1..5]>)
\vspace{0.1in}
\{Exception\} <event> := [datarray.DataArray(psplit(<P>), axes = [<ename>])]
\{Exception\} <ename,1>!=<ename,2> -> <event> := [datarray.DataArray([[psplit(<P>)],psplit(<P>)],[<ename>,<ename>])]
\vspace{0.1in}
<cpts> := []
~<cpts>.append(<event>[0])
\vspace{0.1in}
<evidence> := \{\}
~<evidence>.update([(<ename>,0)])
\vspace{0.1in}
\{Exception\} print(flatten\_and\_sort(inference\_algs.calc\_marginals\_simple(<cpts>,<evidence>)))
\{Exception\} print(flatten\_and\_sort(inference\_algs.calc\_marginals\_sumproduct(<cpts>,<evidence>)))
\{Exception\} print(flatten\_and\_sort(inference\_algs.calc\_marginals\_jtree(<cpts>,<evidence>)))
\end{code}
}
\caption {Complete TSTL harness for finding the hash-order bug in the datarray
  inference algorithms.}
\label{hashbug}
\end{figure*}

\subsection {Vertical Determinism Case Study: {\tt pyfakefs}}

The {\tt pyfakefs} \cite{pyfakefs} module implements a fake file
system that mocks the Python file system modules, to allow Python
tests both to run faster by using an in-memory file system and to make
file system changes that would not be safe or easily performed using
real persistent storage.  Originally developed in 2006 at Google by
Mike Bland, {\tt pyfakefs} is now used in over 2,000 Python tests,
inside and outside Google \cite{pyfakefs}.

The TSTL harness for {\tt pyfakefs} has been used to detect (and
correct) over 50 faults.  However, the testing completely relies on
the existence of a reference file system implementation.

We introduced a subtle bug into {\tt pyfakefs}, where the {\tt remove}
call checks that its target is not a directory, and returns the
correct error, but still carries out the remove operation.  Using {\tt
  os.remove} to delete directories does not break any file system
invariants, but violates the Python {\tt os} specification (and,
indirectly, the usual POSIX implementation behavior where {\tt unlink}
does not work for directories).  Detecting this bug using the TSTL
{\tt pyfakefs} harness is normally impossible without using another
file system as a reference.  However, the fault was detected
immediately, even without using a reference, when we compiled the harness
with the {\tt --checkFailureDeterminism} flag.  Because vertical
reduction does not require running complete tests multiple times, and
does not affect the delta-debugging algorithm's performance, reducing
the failing test to 3 steps required less than a second.
Moreover, the overhead for the check for failure determinism in a
version of the code without the {\tt remove} error was
less than 8\%.  Detecting the fault using a reference file system 
required 17\% more runtime before detection, and took over twice as
long to reduce the failure to a
slightly longer failing test, which did not have {\tt remove} as its
final operation (since further file system interaction is required to
expose the bad state).