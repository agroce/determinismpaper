\section{Conclusions and Future Work}

Unexpected nondeterminism of software systems frustrates users,
whether they be humans or (more importantly) other software systems.
Nondeterminism is even more pernicious in software testing,
frustrating debugging efforts (``Heisenbugs'' are widely loathed \cite{Heisenbug}), and
leading to the costly problem of flaky tests
\cite{miccoflaky,listfieldtestanalysis}.

This paper proposes a formulation of types of nondeterminism, and a
practical approach to using automated test generation to detect
nondeterminism, especially in the library code that underlies most
systems. In addition to traditional \emph{horizontal} nondeterminism
(the phenomenon behind Heisenbugs and flaky tests), we also discuss
the related concept of \emph{vertical} nondeterminism, which is more
frequently simply a software bug.  We implemented our approach in the
TSTL automated test generation system for Python, and demonstrated the
simplicity and basic utility of the approach on three real-world examples.

As future work, we would like to make the automatic detection of
values that should not be visible (e.g., count towards nondeterminism)
possible: if a value (e.g., a timestamp) is always different in every
run, it is likely of no interest to developers.  This would make
testing libraries mixing deterministic behavior and nondeterministic
behavior, e.g. cryptographic libraries offering high-quality random
number generators, easier for users, who would only be shown
``surprising'' nondeterminism.  Similarly, in order to extend the
utility of vertical nondeterminism detection, we would like to
automatically identify usually-idempotent operations, and check them
for deviation from the expected behavior.  We would also like to lower the cost
of checking process nondeterminism, perhaps by using methods taken in
AFL \cite{aflfuzz} to avoid high startup costs for test executions.

More generally, we are interested in using test decomposition
\cite{Composition} to more easily isolate, understand, and avoid
nondeterminism is tests, and to mitigate the problem of flaky tests.
Reliable detection of nondeterminism is a first step towards
evaluating such pro-active flaky test mitigations.

{\scriptsize {\bf Acknowledgments:}  The authors would like to thank John Micco,
Jeff Listfield, and Celal Ziftci at Google, for discussion of
flaky tests, Andreas Zeller and David R. MacIver for discussion of the
problem of probabilistic delta-debugging, and Chris
Colohan for discussion of sources of process-based nondeterminism.}